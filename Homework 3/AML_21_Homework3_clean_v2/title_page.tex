% Cornell CS5785 homework/quiz template.
%
% Original author:
% Frits Wenneker (http://www.howtotex.com)
% Revised by Michael Wilber, Xun Huang
% Revised by Jin Sun, 2020
%
% License:
% CC BY-NC-SA 3.0 (http://creativecommons.org/licenses/by-nc-sa/3.0/)
%
%%%%%%%%%%%%%%%%%%%%%%%%%%%%%%%%%%%%%%%%%

% Title page
\begin{center}
\newcommand{\HRule}{\rule{\linewidth}{0.5mm}}
\HRule \\[0.4cm]
{ \huge \bfseries CS5785 \thetitle} \\ %[0.2cm]
\HRule
\end{center}


\begin{framed}\label{formatting-guidelines}
The homework is generally split into programming exercises and written exercises.

This homework is due on \textbf{\hwduedate}. Upload your homework to Gradescope (Canvas->Gradescope). There are two assignments for this homework in Gradescope. Please note a complete submission should include:
\begin{enumerate}
\item A write-up as a single \texttt{.pdf} file, which should be submitted to "Homework 3 (write-up)" This file should contain your answers to the written questions \textbf{and exported pdf file / structured write-up of your answers to the coding questions} (which should include core codes, plots, outputs, and any comments / explanations). \textbf{We will deduct points if you do not do this}.
\item Source code for all of your experiments (AND figures) zipped into a single .zip file, in \texttt{.py} files if you use Python or \texttt{.ipynb} files if you use the IPython Notebook. If you use some other language, include all build scripts necessary to build and run your project along with instructions on how to compile and run your code. \textbf{If you use the IPython Notebook to create any graphs, please make sure you also include them in your write-up.} This should be submitted to "Homework 3 (code)".
\item You need to mark the pages of your submission to each question on Gradescope after submission, Gradescope should ask you to do that after you upload your write-up by default. \textbf{We will deduct points if you do not do this}.
\end{enumerate}
The write-up should contain a general summary of what you did, how well your solution works, any insights you found, etc. On the cover page, include the class name, homework number, and team member names. You are responsible for submitting clear, organized answers to the questions.
You could use online ~\LaTeX~ templates from \href{https://www.overleaf.com/latex/templates/}{Overleaf}, under ``Homework Assignment'' and and ``Project / Lab Report''. You could also use a \href{https://drive.google.com/file/d/1jn_BEAHw8khTWdTWoU2rQeOxymTnmBSY/view?usp=sharing}{~\LaTeX~ template we made}, which contains useful packages for writing math equations and code snippet. 

Please include all relevant information for a question, including text response, equations, figures, graphs, output, etc. If you include graphs, be sure to include the source code that generated them. Please pay attention to Canvas for relevant information regarding updates, tips, and policy changes. You are encouraged (but not required) to work in groups of 2.
\end{framed}

\section*{If you need help}
There are several strategies available to you.
\begin{itemize}
\item If you get stuck, we encourage you to post a question  on the Discussions section of Canvas. That way, your solutions will be available to other students in the class.
\item The professor and TAs offer office hours, which are a great way to get some one-on-one help.
\item You are allowed to use well known libraries such as \verb+scikit-learn+, \verb+scikit-image+, \verb+numpy+, \verb+scipy+, etc. for this assignment (including implementations of machine learning algorithms), unless we explicitly say that you cannot in a particular question. Any reference or copy of public code repositories should be properly cited in your submission (examples include Github, Wikipedia, Blogs).
\end{itemize}
