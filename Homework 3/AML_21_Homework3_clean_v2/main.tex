% Cornell CS5785 homework/quiz template.
%
% Original author:
% Frits Wenneker (http://www.howtotex.com)
% Revised by Michael Wilber, Xun Huang
% Revised by Jin Sun, 2020
%
% License:
% CC BY-NC-SA 3.0 (http://creativecommons.org/licenses/by-nc-sa/3.0/)
%
%%%%%%%%%%%%%%%%%%%%%%%%%%%%%%%%%%%%%%%%%


\documentclass[letterpaper,12pt]{article}
\usepackage{fourier} % Nicer font
\usepackage{listings}
\usepackage[top=0.7in, bottom=1in, left=0.7in, right=0.7in]{geometry}
\usepackage[english]{babel} % Nicer hyphenation
\usepackage{amsmath,amsfonts,amsthm} % Math packages
\usepackage{sectsty} % Allows customizing section commands
\usepackage[normalem]{ulem}
\usepackage{color}
\usepackage[dvipsnames]{xcolor}
% \newcommand{\todo}{\colorbox{yellow}{\fbox{\LARGE{TODO}}}\vspace{1in}}
\newcommand{\todo}{\colorbox{yellow}{\fbox{\LARGE{TODO}}}}
\newcommand{\blank}[1]{\uline{\phantom{?}\hspace{#1}}}
\newcommand{\truefalse}{\hspace{2ex}\textbf{True}~or~\textbf{False}?}
\usepackage{fancyhdr}
\usepackage{framed}
\usepackage[colorlinks = true,
            linkcolor = blue,
            urlcolor  = blue,
            citecolor = blue,
            anchorcolor = blue]{hyperref}
\pagestyle{fancy} % Use a custom header for all pages...
\thispagestyle{empty} % except the first page
\usepackage{graphicx}
\usepackage[parfill]{parskip}
\usepackage{hyperref}
\usepackage{minted} %highlight code
\newtheorem{theorem}{Theorem}


\newcommand\zhaonotes[1]{\textcolor{orange}{#1}}
\newcommand\vk[1]{\textcolor{red}{#1}}

% display math with correct format
\everymath{\displaystyle}

\definecolor{mygreen}{rgb}{0,0.6,0}
\definecolor{mygray}{rgb}{0.5,0.5,0.5}
\definecolor{mymauve}{rgb}{0.58,0,0.82}
\lstset{ %
  basicstyle=\tt,        % the size of the fonts that are used for the code
  breakatwhitespace=false,         % sets if automatic breaks should only happen at whitespace
  breaklines=true,                 % sets automatic line breaking
  commentstyle=\color{mygreen},    % comment style
  extendedchars=true,              % lets you use non-ASCII characters; for 8-bits encodings only, does not work with UTF-8
  keepspaces=true,                 % keeps spaces in text, useful for keeping indentation of code (possibly needs columns=flexible)
  keywordstyle=\color{blue},       % keyword style
  language=Python,                 % the language of the code
  showspaces=false,                % show spaces everywhere adding particular underscores; it overrides 'showstringspaces'
  showstringspaces=false,
  stepnumber=2,                    % the step between two line-numbers. If it's 1, each line will be numbered
  stringstyle=\color{mymauve},     % string literal style
}


\title{Homework 3}

% Fill in your name and Cornell NetID on the line below:
\author{ Yiran Zhao }

% More junk, skip down to the document body:
\makeatletter
\let\thetitle\@title
\let\theauthor\@author
\fancyhead[R]{Page \thepage} % Set headings and footer
\fancyhead[L]{CS5785 Fall 2021: \@title}
\fancyhead[C]{}
\makeatother
\allsectionsfont{\centering \normalfont\scshape} % Make all sections centered, the default font and small caps
\setlength\parindent{0pt} % Removes all indentation from paragraphs - comment this line for an assignment with lots of text

% Due date for this homework
\newcommand{\hwname}{Homework 3}
\newcommand{\hwduedate}{Oct 14, 2021 at 11:59 PM ET}

\begin{document}

% Title page
% Cornell CS5785 homework/quiz template.
%
% Original author:
% Frits Wenneker (http://www.howtotex.com)
% Revised by Michael Wilber, Xun Huang
% Revised by Jin Sun, 2020
%
% License:
% CC BY-NC-SA 3.0 (http://creativecommons.org/licenses/by-nc-sa/3.0/)
%
%%%%%%%%%%%%%%%%%%%%%%%%%%%%%%%%%%%%%%%%%

% Title page
\begin{center}
\newcommand{\HRule}{\rule{\linewidth}{0.5mm}}
\HRule \\[0.4cm]
{ \huge \bfseries CS5785 \thetitle} \\ %[0.2cm]
\HRule
\end{center}


\begin{framed}\label{formatting-guidelines}
The homework is generally split into programming exercises and written exercises.

This homework is due on \textbf{\hwduedate}. Upload your homework to Gradescope (Canvas->Gradescope). There are two assignments for this homework in Gradescope. Please note a complete submission should include:
\begin{enumerate}
\item A write-up as a single \texttt{.pdf} file, which should be submitted to "Homework 3 (write-up)" This file should contain your answers to the written questions \textbf{and exported pdf file / structured write-up of your answers to the coding questions} (which should include core codes, plots, outputs, and any comments / explanations). \textbf{We will deduct points if you do not do this}.
\item Source code for all of your experiments (AND figures) zipped into a single .zip file, in \texttt{.py} files if you use Python or \texttt{.ipynb} files if you use the IPython Notebook. If you use some other language, include all build scripts necessary to build and run your project along with instructions on how to compile and run your code. \textbf{If you use the IPython Notebook to create any graphs, please make sure you also include them in your write-up.} This should be submitted to "Homework 3 (code)".
\item You need to mark the pages of your submission to each question on Gradescope after submission, Gradescope should ask you to do that after you upload your write-up by default. \textbf{We will deduct points if you do not do this}.
\end{enumerate}
The write-up should contain a general summary of what you did, how well your solution works, any insights you found, etc. On the cover page, include the class name, homework number, and team member names. You are responsible for submitting clear, organized answers to the questions.
You could use online ~\LaTeX~ templates from \href{https://www.overleaf.com/latex/templates/}{Overleaf}, under ``Homework Assignment'' and and ``Project / Lab Report''. You could also use a \href{https://drive.google.com/file/d/1jn_BEAHw8khTWdTWoU2rQeOxymTnmBSY/view?usp=sharing}{~\LaTeX~ template we made}, which contains useful packages for writing math equations and code snippet. 

Please include all relevant information for a question, including text response, equations, figures, graphs, output, etc. If you include graphs, be sure to include the source code that generated them. Please pay attention to Canvas for relevant information regarding updates, tips, and policy changes. You are encouraged (but not required) to work in groups of 2.
\end{framed}

\section*{If you need help}
There are several strategies available to you.
\begin{itemize}
\item If you get stuck, we encourage you to post a question  on the Discussions section of Canvas. That way, your solutions will be available to other students in the class.
\item The professor and TAs offer office hours, which are a great way to get some one-on-one help.
\item You are allowed to use well known libraries such as \verb+scikit-learn+, \verb+scikit-image+, \verb+numpy+, \verb+scipy+, etc. for this assignment (including implementations of machine learning algorithms), unless we explicitly say that you cannot in a particular question. Any reference or copy of public code repositories should be properly cited in your submission (examples include Github, Wikipedia, Blogs).
\end{itemize}

\section*{Programming Exercises}

\begin{enumerate}

\item \textbf{Eigenface for face recognition. (40 pts)}

% \begin{figure}[h]
% \centering
% \includegraphics[width=0.8\textwidth]{hw4/YaleB.jpg}
% \end{figure}

In this assignment you will implement the Eigenface method for recognizing human faces. You will use face images from \textit{The Yale Face Database B}, where there are $64$ images under different lighting conditions per each of $10$ distinct subjects, $640$ face images in total.

\textbf{Read more (optional):}
\begin{itemize}
\item Eigenface on Wikipedia: \url{https://en.wikipedia.org/wiki/Eigenface}
\item Eigenface on Scholarpedia: \url{http://www.scholarpedia.org/article/Eigenfaces}
\end{itemize}

\begin{enumerate}
\item \textbf{(2 pts)} Download \href{http://cornelltech.github.io/cs5785-fall-2019/data/faces.zip}{The Face Dataset} and unzip \verb!faces.zip!,
You will find a folder called \emph{images} which contains all the training and test images; \emph{train.txt} and \emph{test.txt} specifies the training set and test (validation) set split respectively, each line gives an image path and the corresponding label.

\item \textbf{(2 pts)} Load the training set into a matrix $\mathbf{X}$: there are $540$ training images in total, each has $50 \times 50$ pixels that need to be concatenated into a $2500$-dimensional vector. So the size of $\mathbf{X}$ should be $540 \times 2500$, where each row is a flattened face image. Pick a face image from $\mathbf{X}$ and display that image in grayscale. Do the same thing for the test set. The size of matrix $\mathbf{X_{test}}$ for the test set should be $100 \times 2500$.

Below is the sample code for loading data from the training set. You can directly run it in Jupyter Notebook:
\begin{minted}
[frame=lines,
framesep=2mm,
baselinestretch=1.0,
fontsize=\footnotesize,
linenos]
{python}
import numpy as np
from scipy import misc
from matplotlib import pylab as plt
import matplotlib.cm as cm
%matplotlib inline

train_labels, train_data = [], []
for line in open('./faces/train.txt'):
    im = misc.imread(line.strip().split()[0])
    train_data.append(im.reshape(2500,))
    train_labels.append(line.strip().split()[1])
train_data, train_labels = np.array(train_data, dtype=float), np.array(train_labels, dtype=int)

print train_data.shape, train_labels.shape
plt.imshow(train_data[10, :].reshape(50,50), cmap = cm.Greys_r)
plt.show()
\end{minted}

\item \textbf{(3 pts)} Average Face. Compute the \emph{average face} $\mathbf{\mu}$ from the whole training set by summing up every row in $\mathbf{X}$ then dividing by the number of faces. Display the \emph{average face} as a grayscale image.

\item \textbf{(3 pts)} Mean Subtraction. Subtract average face $\mathbf{\mu}$ from every row in $\mathbf{X}$. That is, $\mathbf{x_i} := \mathbf{x_i} - \mathbf{\mu}$, where $\mathbf{x_i}$ is the $i$-th row of $\mathbf{X}$. Pick a face image after mean subtraction from the new $\mathbf{X}$ and display that image in grayscale. Do the same thing for the test set $\mathbf{X_{test}}$ using the pre-computed average face $\mathbf{\mu}$ in (c).


\item \textbf{(10 pts)} Eigenface. Perform eigendecomposition on $\mathbf{X}^T\mathbf{X} = \mathbf{V} \mathbf{\Lambda} \mathbf{V}^T$ to get eigenvectors $\mathbf{V}^T$,
where each row of $\mathbf{V}^T$ has the same dimension as the face image.
We refer to $\mathbf{v_i}$, the $i$-th row of $\mathbf{V}^T$, as $i$-th \emph{eigenface}.
Display the first $10$ eigenfaces as $10$ images in grayscale.

\item \textbf{(10 pts)} Eigenface Feature. The top $r$ eigenfaces $\mathbf{V}^T[:r, :] = \{ v_1, v_2, \dots, v_r \}^T$ span an $r$-dimensional linear subspace of the original image space called \emph{face space}, whose origin is the average face $\mathbf{\mu}$, and whose axes are the eigenfaces $\{ v_1, v_2, \dots, v_r \}$.
Therefore, using the top $r$ eigenfaces $\{ v_1, v_2, \dots, v_r \}$, we can represent a $2500$-dimensional face image $\mathbf{z}$ as an $r$-dimensional feature vector $\mathbf{f}$: $\mathbf{f} = \mathbf{V}^T[:r, :]~ \mathbf{z} = [v_1, v_2, \dots, v_r]^T \mathbf{z}$.
Write a function to generate $r$-dimensional feature matrix $\mathbf{F}$ and $\mathbf{F_{test}}$ for training images $\mathbf{X}$ and test images $\mathbf{X_{test}}$, respectively (to get $\mathbf{F}$, multiply $\mathbf{X}$ to the transpose of first $r$ rows of $\mathbf{V}^T$, $\mathbf{F}$ should have same number of rows as $\mathbf{X}$ and $r$ columns; similarly for $\mathbf{X_{test}}$).

\item \textbf{(10 pts)} \label{part:facerec} Face Recognition. For this problem, you are welcome to use libraries such as \texttt{scikit learn} to perform logistic regression. Extract training and test features for $r=10$. Train a Logistic Regression model using $\mathbf{F}$ and test on $\mathbf{F_{test}}$. Report the classification accuracy on the test set. Plot the classification accuracy on the test set as a function of $r$ when $r = 1, 2, \dots, 200$. Use ``one-vs-rest'' logistic regression, where a classifier is trained for each possible output label. Each classifier is trained on faces with that label as positive data and all faces with other labels as negative data. \verb+sklearn+ calls this the ``ovr'' mode.

\end{enumerate}


\item \textbf{Implement EM algorithm. (40 pts)}
In this problem, you will implement a bimodal GMM model fit using the EM algorithm. Bimodal means that the distribution has two peaks, or that the data is a mixture of two groups. If you want, you can assume the covariance matrix is diagonal (i.e. it has the form $\text{diag}(\sigma_1^2,\sigma_2^2,...,\sigma_d^2)$ for scalars $\sigma_i$) and you can randomly initialize the parameters of the model. 

You will need to use the \href{http://www.stat.cmu.edu/~larry/all-of-statistics/=data/faithful.dat}{Old Faithful Geyser Dataset}. The data file contains 272 observations of the waiting time between eruptions and the duration of each eruption for the Old Faithful geyser in Yellowstone National Park.

You should do this without calling the \href{https://scikit-learn.org/stable/modules/generated/sklearn.mixture.GaussianMixture.html#sklearn.mixture.GaussianMixture}{Gaussian Mixture} library in \texttt{scikit learn}. You can use \texttt{numpy} or \texttt{scipy} for matrix calculation or generating Gaussian distributions. 

\begin{enumerate}
\item (\textbf{2 pts}) Treat each data entry as a 2 dimensional feature vector. Parse and plot all data points on 2-D plane. 

\item (\textbf{3 pts}) Recall that EM learns the parameter $\theta$ of a Gaussian mixture model $P_{\theta}(x, z)$ over a dataset $D = \{x^{(i)} | i = 1, 2, ... n\}$ by performing the E-step and the M-step for $t=0,1,2,\ldots$. We repeat the E-step and and M-step until convergence.

In the E-step, for each $x^{(i)} \in D$, we compute a vector of probabilities $P_{\theta_t}(z = k\mid x)$ for the event that each $x^{(i)}$ originates from a cluster $k$ given the current set of parameters $\theta_{t}$. 

Write the expression for $P_{\theta_t}(z = k\mid x)$, which is the posterior of each data point $x^{(i)}$. Recall that by  Bayes' rule, 
$$
P_{\theta_t}(z = k\mid x) 
= \frac{P_{\theta_t}(z=k, x)}{P_{\theta_t}(x)}
= \frac{P_{\theta_t}(z=k, x)}{\sum_{l = 1}^{K} P_{\theta_t}{(x | z = l)} P_{\theta_t}{(z = l)}}.
$$

Note that we have seen this formula in class. We are asking you to write it down and try to understand and it before implementing it in part (e).

\item (\textbf{5 pts})  In the M-step, we compute new parameters $\theta_{t+1}$. Our goal is to find $\mu_k, \Sigma_k$ and $\phi_{k}$ that optimize
$$\max_\theta \left(\sum_{k = 1}^{K} \sum_{x \in D} P_{\theta_t}(z_{k}|x) \log P_{\theta}(x|z_{k}) +  \sum_{k = 1}^{K} \sum_{x \in D} P_{\theta_t}(z_{k}|x) \log P_\theta(z_{k})\right) $$

Write down the formula for $\mu_k$, $\Sigma_k$, and for the parameters $\phi$ at the M-step (we have also seen these formulas in class).

\item (\textbf{25 pts}) Implement and run the EM algorithm. The points in this question is given by:
\begin{enumerate}
    \item (\textbf{10 pts}) EM implementation. 
    \item (\textbf{5 pts}) You will also need to choose a termination criterion for when the algorithm stops repeating the E-step and the M-step (e.g., a convergence threshold). State your termination criterion and explain the reasoning behind it. 
    \item (\textbf{10 pts}) Plot the trajectories of the two mean vectors ($\mu_1$ and $\mu_2$) in 2 dimensions to show how they change over the course of running EM. You might want to use a scatter plot for this.
\end{enumerate} 

\item (\textbf{5 pts}) If you run $K$-means clustering instead of the EM algorithm you just implemented, do you think you will get different clusters? You are welcome to experiment with $K$-means clustering on the same dataset with $K=2$. (The \href{https://scikit-learn.org/stable/modules/generated/sklearn.cluster.KMeans.html}{K-means} library from \texttt{scikit learn} is a good way to try). Comment on why do you think the results will or will not change. .

\end{enumerate}
\end{enumerate}



\section*{Written Exercises}

\begin{enumerate}

\item \textbf{SVD and eigendecomposition. (10 pts)}
% You implemented a programming version of SVD in the programming section, now let's build a mathematical understanding of it. In the real number case, let's consider a matrix $X$ with $m \times n$ dimensions. 
Recall that the SVD of an $m \times n$ matrix $X$ is the factorization of $X$ into three matrices $X=UDV^{T}$, where $U$ is a $m \times m$ orthonormal matrix, $D$ is a $m \times n$ diagonal matrix with non-negative real numbers on the diagonal, and $V$ is a $n \times n$ orthonormal matrix. An orthonormal matrix just means that $U^{T}U = I$ and $V^{T}V = I$.

Show that we can obtain the eigendecomposition of $X^TX$ from the SVD of a matrix $X$. 


(This tells us that we can do an SVD of $X$ and get same result as the eigendecomposition of $X^TX$ but the SVD is faster and easier.)

% \item \textbf{Weights for clustering.}
% In clustering algorithms like K-means, we need to compute distances in the feature space.
% Sometimes people use weights to value some feature more than others.
% Show that weighted Euclidean distance for $p$ dimensional data points $x_i$ and $x_{i'}$
% \begin{equation}
%     d_e^{(w)}(x_i,x_{i'}) = \frac{\sum_{l=1}^p w_l(x_{il} - x_{i' l})^2}{\sum_{l=1}^p w_l}
% \end{equation}
% satisfies
% \begin{equation}
%     d_e^{(w)}(x_i,x_{i'}) = d_e(z_i,z_{i'}) = \sum_{l=1}^p (z_{il} - z_{i'l})^2,
% \end{equation}
% where
% \begin{equation}
%     z_{il} = x_{il} \cdot \left (\frac{w_l}{\sum_{l=1}^p w_l} \right )^{1/2}.
% \end{equation}
% Thus weighted Euclidean distance based on $x$ is equivalent to unweighted Euclidean distance based on a proper transformed data $z$.

% \item \textbf{SVD of Rank Deficient Matrix. (10 pts)}
% You can use NumPy library for this problem. 
% % it’s an extreme example of dimensionality reduction where the dimensionality d=3 is fully redundant and can be completely captured by p=2 
% In this extreme example of dimensionality reduction, we are going to see the original data's dimensionality is redundant and can be completely captured by lower dimensions.
% Consider matrix $M$. It has rank 2, as you can see by observing that there times the first column minus the other two columns is 0.
% \begin{equation}
%   M = \left[
%     \begin{matrix}
% 1 & 0 & 3 \\
% 3 & 7 & 2 \\
% 2 & -2 & 8 \\
% 0 & -1 & 1 \\
% 5 & 8 & 7
%     \end{matrix}
%   \right] .
% \end{equation}

% \begin{enumerate}
%   \item Compute the matrices $M^TM$ and $MM^T$.
%   \item Find the eigenvalues for your matrices of part (a).
%   \item Find the eigenvectors for the matrices of part (a).
%   \item Find the SVD for the original matrix M from parts (b) and (c). Note that there are only two nonzero eigenvalues, so your matrix $\Sigma$ should have only two singular values, while $U$ and $V$ have only two columns.
%   \item 
% %   Set your smaller singular value to 0 and compute the one-dimensional approximation to the matrix $M$.
%   There are 2 non-zero singular values, if we only keep one by setting the smaller singular value to 0 , then the data will be represented in 1D only. 
%   Compute such one-dimensional approximation to $M$.
% \end{enumerate}


\end{enumerate}



\end{document}